\documentclass[a4paper]{article}

%The following chunk of stuff is to get code snippets in tex

\usepackage{listings}
\usepackage{color}

\definecolor{dkgreen}{rgb}{0,0.6,0}
\definecolor{gray}{rgb}{0.5,0.5,0.5}
\definecolor{mauve}{rgb}{0.58,0,0.82}

\lstset{frame=tb,
  language=Java, %Change language here for different language.
  aboveskip=3mm,
  belowskip=3mm,
  showstringspaces=false,
  columns=flexible,
  basicstyle={\small\ttfamily},
  numbers=none,
  numberstyle=\tiny\color{gray},
  keywordstyle=\color{blue},
  commentstyle=\color{dkgreen},
  stringstyle=\color{mauve},
  breaklines=true,
  breakatwhitespace=true,
  tabsize=3
}

% amsmath package addition
\usepackage{amsmath, hyperref}

%Put title info here
\title{Initial setup for the UR5 communications with ROS}
\author{Angadh Nanjangud}
\date{started on: November 7, 2017\\ edited on: \today}

% Document begins here
\begin{document}
\maketitle

\section{Introduction}
This is a document that will inform the user on using ROS to
communicate with the UR5 on their personal computer. It will
also discuss the usage of MoveIt! for motion planning of the
UR5 and execute said plans on the physical robot.

The pre-requisites are that the user have ROS Kinetic. The
operating system that is being used for this tutorial is
Ubuntu 16.04. YMMV depending on what you have. It is recommended
that the user go through the ROS tutorials available at
\url{wiki.ros.org/tutorials}.

\section{Setting up UR5 to communicate with an Ubuntu machine
via  ROS}
The UR5 is decently documented (but not exceptionally well).
The ROS page on
\href{http://wiki.ros.org/universal_robot/Tutorials/Getting%20Started%20with%20a%20Universal%20Robot%20and%20ROS-Industrial}
{'Getting Started with a Universal Robot and ROS-Industrial'}
is a good starting point. However, I will document the entire
process here for sake of completeness.

\begin{itemize}

  \item {\bf Step 1: Setup a catkin workspace}
    Open up a terminal and create a catkin workspace following
    the instructions
    \href{http://wiki.ros.org/catkin/Tutorials/create_a_workspace}
    {here}. I chose to call the catkin workspace driver\_ws.
    \begin{lstlisting}%Note the bash script choice
    $ mkdir -p ~/driver_ws/src
    $ cd ~/driver_ws
    $ catkin_make
    $ source devel/setup.bash
    \end{lstlisting}
  
  \item{\bf Step 2: Obtain UR5 driver}
    First, acquire the necessary universal robot ROS packages
    by running the following command in your terminal:
    \begin{lstlisting}
    $ sudo apt-get install ros-kinetic-ur-driver ros-kinetic-universal-robot
    \end{lstlisting}
    Since the UR5 robot is running Polyscope 3.x, an additional
    installation of the \texttt{ur\_modern\_driver} is
    necessary which can be obtained via git. Ensure that this
    installation is being done in the previously built
    catkin workspace from step 1; in our case, this is the
    \texttt{~/driver\_ws/src} directory. We then run
    \texttt{catkin\_make} again and also source the
    \texttt{setup.bash}file again:
    \begin{lstlisting}
    $ cd ~/driver_ws/src
    $ git clone https://github.com/Zagitta/ur_modern_driver.git
    $ cd ~/driver_ws/
    $ catkin_make
    $ source devel/setup.bash
    \end{lstlisting}

  \item {\bf Step 3: Connect UR5 and computer}
    Follow instructions on step 3.2 on
    \href{http://wiki.ros.org/universal_robot/Tutorials/Getting%20Started%20with%20a%20Universal%20Robot%20and%20ROS-Industrial}
    {'Getting Started with a Universal Robot and ROS-Industrial'}
    to configure the IP of the robot. This is used in Step 4
    to connect the robot to the dektop computer. In my setup,
    the desktop and robot's CB3 unit are wired to the router.
    The robot's IP has been set to $192.168.0.50$.

  \item{\bf Step 4: Testing the connection}
   In the same terminal where the
    \texttt{source devel/setup.bash} was executed, run the
    following:
    \begin{lstlisting}
    $ roslaunch ur_modern_driver ur5_bringup.launch robot_ip:=192.168.0.50
    \end{lstlisting}
    MAKE SURE NOBODY AND NOTHING IS IN THE VICINTY OF THE ROBOT
    BEFORE RUNNING THIS NEXT STEP. Open a new terminal and run:
    \begin{lstlisting}
    $ rosrun ur_driver test_move.py
    \end{lstlisting}
    You should see the robot come to life at this point.
    \texttt{test\_move.py} is a script within the
    \texttt{ur\_modern\_driver} that we installed from git.
\end{itemize}

\section{MoveIt!, RViz, and the real UR5}
With the driver still running in one of the terminals from step
4 (i.e. \texttt{roslaunch} command), open a new terminal and
run:
\begin{lstlisting}
$ roslaunch ur5_moveit_config ur5_moveit_planning_execution.launch limited:=true
\end{lstlisting}
which basically loads what is needed for motion planning.
In the above snippet, the flag \texttt{limited:=true}
loads limits the joint motions to the range [$-\pi,\pi$].
Then, we launch athe RViz gui with MoveIt! motion planning
plugin by running:

\begin{lstlisting}
$ roslaunch ur5_moveit_config moveit_rviz.launch config:=true
\end{lstlisting}

\section{MoveIt!, RViz, and the UR5 on Gazebo}
Instead of connecting to the real robot as explained in step
4, open a new terminal and run:
\begin{lstlisting}
$ roslaunch ur_gazebo ur5.launch
\end{lstlisting}
and then do the same as mentioned in the previous section.
\begin{lstlisting}
$ roslaunch ur5_moveit_config ur5_moveit_planning_execution.launch limited:=true
\end{lstlisting}
which basically loads what is needed for motion planning.
In the above snippet, the flag \texttt{limited:=true}
loads limits the joint motions to the range [$-\pi,\pi$].
Then, we launch athe RViz gui with MoveIt! motion planning
plugin by running:

\begin{lstlisting}
$ roslaunch ur5_moveit_config moveit_rviz.launch config:=true
\end{lstlisting}


\end{document}
