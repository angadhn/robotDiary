\documentclass[a4paper]{article}

%The following chunk of stuff is to get code snippets in tex

\usepackage{listings}
\usepackage{color}

\definecolor{dkgreen}{rgb}{0,0.6,0}
\definecolor{gray}{rgb}{0.5,0.5,0.5}
\definecolor{mauve}{rgb}{0.58,0,0.82}

\lstset{frame=tb,
  language=Java, %Change language here for different language.
  aboveskip=3mm,
  belowskip=3mm,
  showstringspaces=false,
  columns=flexible,
  basicstyle={\small\ttfamily},
  numbers=none,
  numberstyle=\tiny\color{gray},
  keywordstyle=\color{blue},
  commentstyle=\color{dkgreen},
  stringstyle=\color{mauve},
  breaklines=true,
  breakatwhitespace=true,
  tabsize=3
}

% amsmath package addition
\usepackage{amsmath, hyperref}

%Put title info here
\title{Basic setup for the Robotiq Gripper control
with ROS}
\author{Angadh Nanjangud}
\date{started on: November 17, 2017\\ edited on: \today}

% Document begins here
\begin{document}
\maketitle

\section{Robotiq ROS-I Repository}
As with anything that needs to be build in ROS, I took the 
I cloned the git robotiq repository from
\href{https://github.com/ros-industrial/robotiq}{here}.
However, subsequent to cloning, I had issues with building
(i.e. run \texttt{catkin\_make}) as the robotiq repo is for
ROS Jade whereas we are running Kinetic. These were resolved
via conversations with the developer and just putzing around
based on errors. The conversation and fixes are discussed
\href{https://github.com/ros-industrial/robotiq/issues/116}
{here}. My kinetic-devel branch is available
\href{https://github.com/angadhn/robotiq/tree/kinetic-devel}
{here}.

\section{Setting up UR5 to communicate with an Ubuntu machine
via  ROS}
The Robotiq Gripper setup is not well documented. Or, rather,
I should say that the documentation required som significant
hunting. Helpful resources in getting this to work specifically
can be found
\href{https://us.v-cdn.net/6027406/uploads/editor/g9/o29lnir20s3t.pdf}
{here}(the pdf was found via some hunting around on
Robotiq's DoF community).

First, open a new terminal and start $roscore$.
Then, open another terminal and run the following commands 
to find out the port on which the controller is connected,
and to change the access permissions on the USB port. And
then start the driver's node. The last step assumes you have
already sourced \texttt{setup.bash} from the \texttt{devel}
directory of the workspace to which the repository was cloned.
\begin{lstlisting}

$ source kineticDevelWS/devel/setup.bash
$ dmesg | grep tty
$ sudo chmod a+wr /dev/ttyACM0sudo 
$ rosrun robotiq\_c\_model\_control CModelRtuNode.py /dev/ttyACM0
\end{lstlisting}

The driver listens for messages on ``CModelRobotOutput''
using the ``SModel\_robot\_output'' msg type. The messages
are interpreted and commands are sent to the Gripper accordingly.
A simple controller node is provided which can be run in
another terminal, using the command:
\begin{lstlisting}
$ rosrun robotiq\_c\_model\_control CModelSimpleController.py
\end{lstlisting}
They are definitely a tad buggy in that the
\texttt{/dev/ttyACM0} does not show up sometimes. I will
have to examine that some more and update this document
once I have it all figure out.

\section{Wrist Camera introduction}
The gripper is mounted on the Robotiq wrsit cam and was detectable on
Ubuntu under \texttt{dev}; this implies that you can stream the data.
ROS provides many packages that can access and  publish the data. Based
on Pete's advice, I installed the binaries for
\href{http://wiki.ros.org/usb\_cam}{usb\_cam} via apt for ROS
kinetic: \texttt{sudo apt-get install ros-kinetic-usb-cam}. Then,
you should be able to launch a node for the cam by:
\texttt{roslaunch usb\_cam usb\_cam\-test.launch /dev/ttyACM0}.

\section{Useful things to read}
[1] 
[2] \href{https://blog.robotiq.com/bid/58716/Robotiq-s-Adaptive-Robot-Gripper-Now-on-ROS-Industrial}{Robotiq blog on ROS gripper}


\end{document}
